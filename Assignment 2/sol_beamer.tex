%%%%%%%%%%%%%%%%%%%%%%%%%%%%%%%%%%%%%%%%%%%%%%%%%%%%%%%%%%%%%%%
%
% Welcome to Overleaf --- just edit your LaTeX on the left,
% and we'll compile it for you on the right. If you open the
% 'Share' menu, you can invite other users to edit at the same
% time. See www.overleaf.com/learn for more info. Enjoy!
%
%%%%%%%%%%%%%%%%%%%%%%%%%%%%%%%%%%%%%%%%%%%%%%%%%%%%%%%%%%%%%%%


% Inbuilt themes in beamer
\documentclass{beamer}

\def\inputGnumericTable{}

\usepackage[latin1]{inputenc}                                 %%
\usepackage{color}                                            %%
\usepackage{array}                                            %%
\usepackage{longtable}                                        %%
\usepackage{calc}                                             %%
\usepackage{multirow}                                         %%
\usepackage{hhline}                                           %%
\usepackage{ifthen}                                           %%


% Theme choice:
\usetheme{CambridgeUS}

% Title page details: 
\title{Assignment 2 \\ ICSE 2018 Class 12 Q.5(a)} 
\author{Kota Vignan - CS21BTECH11029}
\date{\today}
\logo{\large \LaTeX{}}

\providecommand{\brak}[1]{\ensuremath{\left(#1\right)}}

\begin{document}

% Title page frame
\begin{frame}
    \titlepage 
\end{frame}

% Remove logo from the next slides
\logo{}


% Outline frame
\begin{frame}{Outline}
    \tableofcontents
\end{frame}


\section{Question}
\begin{frame}{Question}
Show that the function,
\begin{equation*}
        f(x) = \begin{cases}
                       x^2, & x\le 1 \\ 
                       \frac{1}{x}, & x > 1
               \end{cases}
\end{equation*}                     
is continuous at $ x = 1 $ but not differentiable.
\end{frame}

 
\section{Solution}
\begin{frame}{Solution}
Given  
            \begin{equation*} 
                 f\brak{x}  = \begin{cases}
                              x^2,  & x \leq 1 \\
                              \frac{1}{x} , & x  >  1
                           \end{cases}
            \end{equation*}  
   We can say f is continuous at $x = 1$,iff 
            \begin{align}
                 \lim_{x \to 1}f\brak{x} &= f\brak{1}
            \end{align}
    In other words f should satisfy,
            \begin{align}
                 f\brak{1^-} =f\brak{1^+} =f\brak{1} 
            \end{align}
    where,
            \begin{align}
                 f\brak{1^-} &=  \lim_{h\to0}f\brak{1-h}\\
                 f\brak{1^+} &=  \lim_{h\to0}f\brak{1+h}
            \end{align}
            \begin{align}
                 f\brak{1} &= 1\label{eq 5}
            \end{align}
\end{frame}


\section{Continuity at $x=1$}
\begin{frame}{Continuity at $x=1$}
Now,
            \begin{align}
                 f\brak{1^-}  &= \lim_{h\to 0}f\brak{1-h} \\
                              &= \lim_{h\to0}\brak{1-h}^2 \\
                 \implies f\brak{1^-} &= 1
            \end{align}
And,
            \begin{align}
                 f\brak{1^+} &= \lim_{h\to0}f\brak{1+h} \\
                             &= \lim_{h\to0}\frac{1}{\brak{1+h}} \\
                 \implies f\brak{1^+} &= 1
            \end{align}
    Using eq 5 ,eq 8,eq 11,we can say that f is continuous at $x = 1$.
\end{frame}


\section{Differentiability at $x=1$}
\begin{frame}{Differentiability at $x=1$}
We can say that f is differentiable at $ x = 1 $ iff the limit,
             \[
                lim_{h \to 0}\frac{f\brak{1 + h} - f\brak{1}}{h}
             \] exists.\\
    In that case f should satisfy,
            \begin{align}
                 \lim_{h \to 0}\frac{f\brak{1 + h} - f\brak{1}}{h}  &=  lim_{h \to 0}\frac{f\brak{1} - f\brak{1-h}}{h} % LHD = RHD
            \end{align}
\end{frame}

\section{LHD}
\begin{frame}{LHD}
            \begin{align}
                 \lim_{h \to 0}\frac{f\brak{1 + h} - f\brak{1}}{h} & = \lim_{h \to 0}\frac{\brak{\frac{1}{\brak{1+h}}} - 1}{h}\\
                 & = \lim_{h \to 0}\frac{{\brak{1 - \brak{1 + h}}}}{h\brak{1 + h}}\\
                 & = \lim_{h \to 0}\frac{-h}{h\brak{1 + h}}\\                                       
                 & = \lim_{h \to 0}\frac{-1}{\brak{1 + h}}\\                         
                 & = -1\\
                 \implies \lim_{h \to 0}\frac{f\brak{1 + h} - f\brak{1}}{h} & = -1.
            \end{align}
\end{frame}


\section{RHD}
\begin{frame}{RHD}
\textbf{RHS:}
            \begin{align}
                 \lim_{h \to 0}\frac{f\brak{1} - f\brak{1-h}}{h}  &= \lim_{h \to 0}\frac{1 -\brak{  \brak{1 -h}^2}}{h}\\
                                                                                      &= \lim_{h \to 0}\frac{-\brak{2h - h^2}}{h}\\
                    &= \lim_{h \to 0}\frac{h\brak{2 - h}}{h}\\                                       
                &= \lim_{h \to 0}-\brak{2 - h}\\    & = 2\\
                 \implies  \lim_{h \to 0}\frac{f\brak{1} - f\brak{1-h}}{h} &= 2.\\
                 \therefore LHS &\neq RHS \nonumber
            \end{align}
    Hence, function $f(x)$ is not differentiable at $ x = 1 $.
\end{frame}


\section{Conclusion}
\begin{frame}{Conclusion}
Therefore, we proved that  $ f\brak{x} = \begin{cases}
                                         x^2, & x \leq 1 \\ 
                                        \frac{1}{x}, & x > 1
                                        \end{cases} $ is continuous at $ x = 1 $ but not differentiable. 
\end{frame}


\end{document}